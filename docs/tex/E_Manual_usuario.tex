\apendice{Documentación de usuario}

\section{Introducción}
 En esta sección se presenta un breve manual de usuario que permita interaccionar con la herramienta de manera satisfactoria, aprovechando al máximo sus características y funcionalidades.

\section{Requisitos de usuarios}
 En este apartado se enumerarán los requisitos que el usuario debe cumplir para poder interactuar con la aplicación satisfactoriamente. 
\begin{itemize}
\tightlist
    \item Sistema Operativo: Indiferente, ya que el usuario únicamente interactúa con la aplicación web donde se encuentra alojado chatbot.
    \item Navegador web: Se recomienda a los usuarios utilizar un navegador web actualizado, como Google Chrome o Mozilla Firefox. Esto asegurará una mejor compatibilidad con las tecnologías utilizadas en la aplicación y ofrecerá una experiencia de usuario óptima.    
    \item Conexión a Internet: Para utilizar la aplicación web, los usuarios deben tener acceso a una conexión estable a Internet. Esto permitirá la comunicación con el servidor donde se encuentra alojada la aplicación web y las aplicaciones en la nube. 
\end{itemize}
 
\section{Instalación}
Dado que se trata de una aplicación web, el usuario no necesita instalar ningún \textit{software} en su dispositivo. Bastará con acceder (asegurándose siempre de tener conexión a Internet) a la dirección donde se aloja la aplicación: \url{https://aws-gcp-chatbot-webapp.s3.amazonaws.com/index.html}.
\section{Manual del usuario}

\subsection{Página principal}
La página principal es la página del comienzo por defecto, donde accederá en primer lugar el usuario. Aquí podrá leer una breve descripción del funcionamiento de la aplicación, así como de las dos plataformas implementadas. En la barra superior, podrá acceder a cualquiera de las dos.
\imagen{Paginaprincipal.png}{Pagina principal de la webapp}{0.9}

\subsection{Página AWS}
En esta página el usuario accede para interactuar con el bot de AWS. En la parte inferior derecha de la pantalla, le saldrá un desplegable donde podrá interactuar con el bot.
\imagen{AWS_webapp.png}{Página AWS de la webapp}{0.5}

\subsection{Página GCP}
En esta página el usuario accede para interactuar con el bot de GCP. En la parte inferior derecha de la pantalla, le saldrá un desplegable donde podrá interactuar con el bot.

\imagen{GCP_webapp.png}{Página GCP de la webapp}{0.5}

\apendice{Plan de Proyecto Software}

\section{Introducción}
La planificación temporal es esencial para el éxito de cualquier proyecto, especialmente
en el desarrollo de \textit{software}. Esta se puede definir como una etapa que implica
detallar los objetivos y el alcance, identificar los requisitos y recursos
necesarios, establecer un cronograma con hitos clave, asignar tareas a los
miembros del equipo y prever los posibles riesgos.

En la memoria se ha indicado que se ha seguido una metodología ágil de gestión
de proyectos, con clara fundamentación en Scrum\citep{wiki:scrum}.

En esta sección se detallará cómo se ha llevado a cabo el cronograma siguiendo esa
metodología, mediante el uso de \textit{sprints}. Se describirán los pasos
necesarios para planificar y gestionar de manera eficiente el tiempo y los
recursos disponibles, asegurando así el cumplimiento de los objetivos del proyecto
en el plazo establecido.

Algunos de los aspectos más relevantes que han cubierto esta filosofía han sido:

\begin{itemize}
	\tightlist

	\item Desarrollo incremental a través de iteraciones llamadas \emph{sprints}.

	\item Utilización de repositorio \emph{Git}\citep{online:git} para solicitud
		de mejoras y para realizar un seguimiento de la evolución, con acceso para los
		tutores desde las fases iniciales.

	\item Control temporal a través de los \emph{sprints}, valorando al inicio de cada
		uno cuál sería la duración adecuada basándonos en las tareas que se iban a realizar
		y el producto que se esperaba al final de esa iteración.

\end{itemize}

Además de los \emph{sprints} se diseñaron hitos relevantes denominados \emph{Milestones}
en \emph{GitHub} que sirven como referencia de la producción realizada.

\section{Metodología Scrum}

La metodología Scrum es un marco de trabajo ágil que se utiliza para la gestión
de proyectos. Se basa en la colaboración entre el equipo de desarrollo y el cliente,
así como en la entrega continua de productos funcionales en ciclos cortos de
tiempo conocidos como \textit{sprints}~\cite{Palacio2022}. Scrum se centra en la
flexibilidad y la adaptabilidad, facilitando los reajustes y los cambios ante situaciones
inesperadas.

El uso de esta metodología permitirá tener un enfoque más dinámico y flexible,
permitiendome así el adaptarme a las necesidades a lo largo del proyecto. Además,
Scrum proporcionará una mayor transparencia y comunicación con el tutor, lo que nos capacitará
para tomar decisiones informadas y asegurar que el proyecto se entregue a tiempo
y con éxito.

A continuación, se presentarán algunos conceptos clave que son fundamentales
para comprender la metodología Scrum.

\subsection{Roles en Scrum}

En Scrum, existen tres roles principales: el Scrum Master, el Product Owner y el
Equipo de Desarrollo.
\begin{enumerate}
	\item El Scrum Master es responsable de garantizar que el equipo siga las prácticas
		y principios de Scrum, eliminando cualquier obstáculo que pueda afectar a la
		productividad.

	\item El Product Owner es el encargado de definir y priorizar los elementos del
		\textit{product backlog}, asegurándose de que se cumplan las necesidades del
		cliente.

	\item Por último, el Equipo de Desarrollo es responsable de la implementación
		de las tareas y la entrega del producto.
\end{enumerate}

Dado que en este caso el equipo está conformado solamente por el tutor y alumno,
es importante adaptar los roles y las prácticas de Scrum para que se ajusten a esta
situación específica. Se pueden asignar los roles de Scrum Master y Product
Owner al propio tutor y, el rol de Equipo de Desarrollo, correspondería al
alumno.

\subsection{Medición de las tareas}

Scrum utiliza la técnica de medición de tareas para estimar el esfuerzo relativo
requerido para completar una tarea. Los \textit{story points} son una medida abstracta
que no se relaciona directamente con el tiempo, sino con la complejidad y el
esfuerzo necesario para completar una tarea en comparación con otras. Esta medida
ayuda al equipo a planificar su capacidad de trabajo en cada \textit{sprint} y a
realizar estimaciones más precisas en función de su experiencia previa. Sin embargo, en este proyecto no se han utilizado.

\subsection{Artefactos en Scrum}

Se denomina Artefacto a aquellos elementos físicos que se producen como resultado
de la aplicación de Scrum. Hay tres principales: el \textit{product backlog}, el
\textit{sprint backlog} y el incremento.
\begin{enumerate}
	\item El \textit{product backlog} es una lista priorizada de todas las
		funcionalidades, características o mejoras que se desean implementar en el producto
		final.

	\item El \textit{sprint backlog} es una selección de elementos del \textit{backlog}
		del producto que se abordarán en un \textit{sprint} específico.

	\item Y finalmente, el incremento es el resultado tangible y potencialmente
		entregable al final de cada \textit{sprint}, que incorpora nuevas
		funcionalidades o mejoras al producto.
\end{enumerate}

\section{Herramienta}
Para llevar a cabo este proyecto, se ha utilizado GitHub como herramienta colaborativa.
GitHub proporciona un entorno seguro y eficiente para almacenar y gestionar el
código fuente, permitiendo trabajar de manera transparente y controlada.

\section{Planificación temporal}

El presente proyecto comenzó en febrero y se extendió hasta Julio. Durante este
período de tiempo, ha sido importante asegurarnos de que todas las tareas e hitos
hayan estado claramente definidos.

Para lograr esto, se han realizado reuniones periódicas para discutir el progreso
del proyecto y asegurarnos de estar en el camino correcto. Además, se han implementado
\textit{sprints} regulares (de entre una y dos semanas de duración) para
asegurarnos de que estamos cumpliendo con nuestras metas a tiempo.

Con esta planificación temporal sólida, hemos podido estar seguros de completar
el proyecto a tiempo y cumplir con los objetivos establecidos. Sin embargo, es importante
ser flexibles y estar preparados para hacer ajustes según sea necesario a medida
que avanzamos en el proyecto, puesto que se realiza al mismo tiempo que
trascurre el curso académico.

En las siguientes tablas se recogen los distintos \textit{sprints} junto con su duración
y objetivos. La descripción detallada de las tareas y su duración puede verse en
el repositorio de GitHub.

\subsubsection{Fase Previa}

\begin{table}[h]
	\centering
	\begin{tabularx}
		{\textwidth}{l X X} \toprule \textbf{Sprint} & \textbf{Fecha} & \textbf{Objetivo}
		\\ \midrule N/A & Diciembre de 2023 & Elegir el proyecto \\ \bottomrule
	\end{tabularx}
	\caption{Detalles de la Fase Previa}
	\label{tab:fase-previa}
\end{table}

\textbf{Tareas:}
\begin{itemize}
	\item Idea de TFG
\end{itemize}

\textbf{Contexto:} Inicio del último curso del Grado de Ingeniería Informática.
Se trata de contactar con los posibles tutores que controlen el desarrollo de un
TFG. Este TFG estará basado en la creación de dos chatbot utilizando los servicios
de AWS y GCP. Esta fase no ha sido establecida como \textit{sprint} en Github pero
se considera parte del proyecto.

\subsubsection{Fase Inicial}

\begin{table}[h]
	\centering
	\begin{tabularx}
		{\textwidth}{l X X} \toprule \textbf{Sprint} & \textbf{Fecha} & \textbf{Objetivo}
		\\ \midrule 0 & 18/03/2024 - 01/04/2024 & Comenzar el desarrollo del proyecto
		\\ \bottomrule
	\end{tabularx}
	\caption{Detalles del Sprint 0}
	\label{tab:sprint0}
\end{table}

\textbf{Tareas:}
\begin{itemize}
	\item Elegir un modelo de referencias bibliográficas

	\item Definición de conceptos
\end{itemize}

\textbf{Contexto:} Se trata de la fase inicial del proyecto, donde se establecen
las bases y se comienzan a definir los conceptos clave.

\subsubsection{Sprint 1}

\begin{table}[h]
	\centering
	\begin{tabularx}
		{\textwidth}{l X X} \toprule \textbf{Sprint} & \textbf{Fecha} & \textbf{Objetivo}
		\\ \midrule 1 & 08/04/2024 - 22/04/2024 & Labor de investigación en AWS \\ \bottomrule
	\end{tabularx}
	\caption{Detalles del Sprint 1}
	\label{tab:sprint1}
\end{table}

\textbf{Tareas:}
\begin{itemize}
	\item Búsqueda de servicios en AWS

	\item Comprensión del funcionamiento de estos servicios
\end{itemize}

\textbf{Contexto:} Esta fase puedo decir que fue la más complicada de todas. Consistió
en investigar acerca de la plataforma AWS, todos los servicios que proporciona y
que me podían ser útiles, así como búsqueda de documentación, ensayos, tutoriales,
etc. Tras este sprint, no solo salí bastante capacitado para empezar a trabajar
en AWS, sino también en GCP dada la similitud en la forma de trabajar en multitud de servicios.

\subsubsection{Sprint 2}

\begin{table}[h]
	\centering
	\begin{tabularx}
		{\textwidth}{l X X} \toprule \textbf{Sprint} & \textbf{Fecha} & \textbf{Objetivo}
		\\ \midrule 2 & 22/04/2024 - 28/04/2024 & Labor de investigación en GCP \\ \bottomrule
	\end{tabularx}
	\caption{Detalles del Sprint 2}
	\label{tab:sprint2}
\end{table}

\textbf{Tareas:}
\begin{itemize}
	\item Equivalencia de servicios de AWS en GCP

	\item Comprensión del funcionamiento de estos servicios
\end{itemize}

\textbf{Contexto:} Este sprint tuvo un objetivo similar al anterior. Tras tener el
grupo de servicios de AWS que podían satisfacer mis necesidad, busqué su
equivalencia en GCP. Necesitaba encontrar aquellos que cumplieran con las mismas
necesidades, además de comprenderlos. La mayoría de ellos obtenían el mismo fin,
pero con diferentes formas de usarlos.

\subsubsection{Sprint 3}

\begin{table}[h]
	\centering
	\begin{tabularx}
		{\textwidth}{l X X} \toprule \textbf{Sprint} & \textbf{Fecha} & \textbf{Objetivo}
		\\ \midrule 3 & 29/04/2024 - 04/05/2024 & Prototipo de AWS \\ \bottomrule
	\end{tabularx}
	\caption{Detalles del Sprint 3}
	\label{tab:sprint3}
\end{table}

\textbf{Tareas:}
\begin{itemize}
	\item Comenzar a trabajar con los servicios

	\item Realizar un primer prototipo del entorno en AWS
\end{itemize}

\textbf{Contexto:} Este sprint consistió en la creación del primer entorno en AWS
que permitiera la extracción de texto desde los PDF para formato TXT utilizando los
servicios de la plataforma.

\subsubsection{Sprint 4}

\begin{table}[h]
	\centering
	\begin{tabularx}
		{\textwidth}{l X X} \toprule \textbf{Sprint} & \textbf{Fecha} & \textbf{Objetivo}
		\\ \midrule 4 & 13/05/2024 - 19/05/2024 & Prototipo de GCP \\ \bottomrule
	\end{tabularx}
	\caption{Detalles del Sprint 4}
	\label{tab:sprint4}
\end{table}

\textbf{Tareas:}
\begin{itemize}
	\item Comenzar a trabajar con los servicios

	\item Realizar un primer prototipo del entorno en GCP
\end{itemize}

\textbf{Contexto:} El sprint 4 tuvo como objetivo desarrollar el mismo entorno
realizado en el sprint 3 pero utilizando los servicios de GCP.

\subsubsection{Sprint 5}

\begin{table}[h]
	\centering
	\begin{tabularx}
		{\textwidth}{l X X} \toprule \textbf{Sprint} & \textbf{Fecha} & \textbf{Objetivo}
		\\ \midrule 5 & 20/05/2024 - 02/06/2024 & Prototipo de función lambda AWS \\
		\bottomrule
	\end{tabularx}
	\caption{Detalles del Sprint 5}
	\label{tab:sprint5}
\end{table}

\textbf{Tareas:}
\begin{itemize}
	\item Crear una lambda para la interacción con el bot

	\item Creación de intents y slots en Lex
\end{itemize}

\textbf{Contexto:} Este sprint fue uno de los mas importantes ya que empecé a crear
el objetivo de este proyecto: el chatbot. Cree el primer prototipo de función lambda
que interacturara con el bot para responder preguntas y explicar conceptos de creación
del chatbot.

\subsubsection{Sprint 6}

\begin{table}[h]
	\centering
	\begin{tabularx}
		{\textwidth}{l X X} \toprule \textbf{Sprint} & \textbf{Fecha} & \textbf{Objetivo}
		\\ \midrule 6 & 10/06/2024 - 24/06/2024 & Prototipo de cloud function \\ \bottomrule
	\end{tabularx}
	\caption{Detalles del Sprint 6}
	\label{tab:sprint6}
\end{table}

\textbf{Tareas:}
\begin{itemize}
	\item Crear una cloud function para la interacción con el bot

	\item Creación de intents y slots en dialogflow
\end{itemize}

\textbf{Contexto:} En este sprint se realizó lo equivalente de AWS del sprint
anterior pero en GCP, con la novedad de la implementación de una lógica para la BBDD
noSQL que permita recoger respuestas a las preguntas, así como los intents
correspondientes en GCP.

\subsubsection{Sprint 7}

\begin{table}[h]
	\centering
	\begin{tabularx}
		{\textwidth}{l X X} \toprule \textbf{Sprint} & \textbf{Fecha} & \textbf{Objetivo}
		\\ \midrule 7 & 24/06/2024 - 09/07/2024 & Comenzar con la memoria \\ \bottomrule
	\end{tabularx}
	\caption{Detalles del Sprint 7}
	\label{tab:sprint7}
\end{table}

\textbf{Tareas:}
\begin{itemize}
	\item Creación de toda la documentación del proyecto

	\item Implementación BBDD AWS

	\item Corrección de errores

	\item Prototipo webapp
\end{itemize}

\textbf{Contexto:} Este es el último sprint, donde ya casi todo el trabajo en
ambas plataformas está hecho y toca empezar a documentar en la memoria todo el
proyecto realizado. También se tiene que implementar la BBDD realizada en el sprint
anterior pero en AWS, así como una webapp donde implementar los chatbot.

\section{Estudio de viabilidad}

El estudio de viabilidad es un componente esencial en el proceso de evaluación de
cualquier proyecto o iniciativa. Dado que la intención no es obtener beneficios
económicos directos de este proyecto, se considerará un enfoque empresarial realista
para asegurar la viabilidad y sostenibilidad a largo plazo de nuestro proyecto.

En las siguientes secciones, se analizarán en detalle los aspectos de viabilidad
económica (sección \ref{sec:Viabilidad económica}) y legal (sección \ref{sec:Viabilidad legal}), con el objetivo de establecer un marco sólido que asegure la continuidad
y éxito del proyecto.

\subsection{Viabilidad económica}\label{sec:Viabilidad económica}
La viabilidad económica se refiere a la evaluación de los recursos financieros necesarios
para llevar a cabo el proyecto y asegurar su sostenibilidad a largo plazo.

Es importante considerar los costos asociados al desarrollo, mantenimiento y
mejora continua de la plataforma, así como analizar posibles fuentes de financiación,
como subvenciones, donaciones o colaboraciones con instituciones interesadas en
respaldar este tipo de iniciativas científicas abiertas.

Vamos a considerar una empresa que enseñe a las demás empresas a crear chatbots utilizando
los servicios de AWS y GCP.

Algunos aspectos relevantes para poder comenzar a hacer estos cálculos son los siguientes:
\begin{itemize}
	\item La empresa se situaría en España, esto implica tener en cuenta las
		regulaciones y requisitos legales, además de fiscales del país.

	\item El proyecto tendría una duración estimada de 4 meses, desde marzo hasta
		julio. Esto equivaldría aproximadamente a 16 semanas, considerando una
		duración promedio de 4 semanas por mes.

	\item El equipo encargado del proyecto estaría formado por un desarrollador (el
		alumno), así como por el \textit{Product Owner} y el \textit{Scrum Master}, quienes
		ambos serían el tutor del alumno.
\end{itemize}

Para todo esto, es fundamental considerar los costes y beneficios asociados al
proyecto:
\begin{itemize}
	\item Los \textbf{costes} se refieren a los desembolsos monetarios necesarios
		para llevar a cabo la iniciativa, incluyendo los gastos de inversión, los
		costes operativos y los costes de mantenimiento~\cite{costebeneficio2006}.

	\item Los \textbf{beneficios} se refieren a las ganancias económicas que se
		esperan obtener como resultado de la implementación del proyecto. Estos beneficios
		pueden incluir ingresos generados por la venta de productos o servicios,
		ahorros en costes operativos, incremento en la productividad, entre otros~\cite{costebeneficio2006}.
\end{itemize}

\subsubsection{Costes}
A continuación, se desarrollan los costes totales de la empresa:

\begin{enumerate}
	\item \underline{Costes de software}.

		Comenzaremos por contemplar las licencias correspondientes a los sistemas
		operativos y herramientas utilizadas.

		\begin{itemize}
			\item {Windows 11 Pro}: El valor aproximado de la licencia de Windows 11
				Pro 159 USD. Con una vida útil de 4 años:

				\[
					\text{{Coste anual de amortización}}= \frac{{159 \,\text{{USD}}}}{4 \,\text{{años}}}
					\approx 40 \,\text{{USD/año}}
				\]
				La duración del proyecto ha sido de 4 meses, por lo que los costes en
				proporción a estos 4 meses suponen un 33,3\%

				\[
					\text{{Proporción de costes}}= \frac{4 \,\text{{meses}}}{12 \,\text{{meses (año)}}}
					\approx 0.33 \, \text{(33.3\%)}
				\]

				Aplicando el 33,3\% del porcentaje anual que ha supuesto el proyecto ,
				tenemos lo siguiente:

				\[
					\text{{Costes de licencia Windows}}= 40 \,\text{{USD}}\times 0.33 \approx
					13.2 \,\text{{USD}}
				\]
		\end{itemize}

		Se adjunta una estimación de los posibles costes en el uso de los servicios para
		crear el chatbot tanto en AWS como en GCP.

		\begin{enumerate}
			\item \underline{Amazon Web Services}

				\begin{itemize}
					\item \textbf{AWS Lambda}: \$0.20 USD por cada millón de solicitudes.

					\item \textbf{Amazon S3}:
						\begin{itemize}
							\item Solicitudes de introducir u obtener documentos: \$0.00045 USD
								por cada 1000 solicitudes.

							\item Almacenamiento: \$0.023 USD por GB para los primeros 50 TB/mes.
						\end{itemize}

					\item \textbf{Amazon Textract}: \$1.50 USD por cada 1000 páginas.

					\item \textbf{Amazon SNS}: \$0.60 USD por millón de notificaciones.

					\item \textbf{Amazon Comprehend}: \$0.0001 USD por unidad hasta 10
						millones de unidades (1 unidad = 100 caracteres).

					\item \textbf{Amazon Lex}: \$0.75 USD por cada 1000 solicitudes.

					\item \textbf{Amazon DynamoDB}:
						\begin{itemize}
							\item Escrituras: \$1.25 USD por millón de escrituras.

							\item Lecturas: \$0.25 USD por millón de lecturas.
						\end{itemize}

					\item \textbf{Amazon Translate}: \$15.00 USD por millón de caracteres.
				\end{itemize}

				\begin{table}[h]
					\centering
					\begin{tabularx}
						{\textwidth}{@{}lX@{}} \toprule \textbf{Servicio AWS} & \textbf{Coste
						por Solicitud/Unidad/Almacenamiento} \\ \midrule AWS Lambda & \$0.20
						USD por millón de solicitudes \\ Amazon S3 &
						\begin{tabular}[t]{@{}l@{}}
							• \$0.00045 USD por 1000 solicitudes             \\
							• \$0.023 USD por GB para los primeros 50 TB/mes
						\end{tabular}
						\\ Amazon Textract & \$1.50 USD por 1000 páginas \\ Amazon SNS & \$0.60
						USD por millón de notificaciones \\ Amazon Comprehend & \$0.0001 USD
						por unidad (100 caracteres) \\ Amazon Lex & \$0.75 USD por 1000 solicitudes
						\\ Amazon DynamoDB &
						\begin{tabular}[t]{@{}l@{}}
							• \$1.25 USD por millón de escrituras \\
							• \$0.25 USD por millón de lecturas
						\end{tabular}
						\\ Amazon Translate & \$15.00 USD por millón de caracteres \\
						\bottomrule
					\end{tabularx}
					\caption{Costes por Solicitud/Unidad/Almacenamiento de los servicios
					AWS utilizados en el proyecto}
					\label{tab:costes_aws}
				\end{table}

				Vamos a asumir las siguientes cifras de uso mensual:

				\begin{itemize}
					\item \textbf{AWS Lambda}: 5 millones de solicitudes.

					\item \textbf{Amazon S3}:
						\begin{itemize}
							\item Solicitudes: 500,000 solicitudes de introducir u obtener
								datos.

							\item Almacenamiento: 100 GB.
						\end{itemize}

					\item \textbf{Amazon Textract}: 5,000 páginas.

					\item \textbf{Amazon SNS}: 2 millones de notificaciones.

					\item \textbf{Amazon Comprehend}: 5,000 unidades.

					\item \textbf{Amazon Lex}: 10,000 solicitudes de texto.

					\item \textbf{Amazon DynamoDB}:
						\begin{itemize}
							\item Escrituras: 3.55 millones de escrituras.

							\item Lecturas: 3.55 millones de lecturas.
						\end{itemize}

					\item \textbf{Amazon Translate}: 1 millón de caracteres.
				\end{itemize}

				\begin{table}[h]
					\centering
					\begin{tabularx}
						{\textwidth}{@{}lX@{}} \toprule \textbf{Servicio AWS} & \textbf{Uso
						Mensual Estimado} \\ \midrule AWS Lambda & 5 millones de solicitudes
						\\ Amazon S3 &
						\begin{tabular}[t]{@{}l@{}}
							• 500,000 solicitudes \\
							• 100 GB
						\end{tabular}
						\\ Amazon Textract & 5,000 páginas \\ Amazon SNS & 2 millones de
						notificaciones \\ Amazon Comprehend & 5,000 unidades \\ Amazon Lex &
						10,000 solicitudes de texto \\ Amazon DynamoDB &
						\begin{tabular}[t]{@{}l@{}}
							• 3.55 millones de escrituras \\
							• 3.55 millones de lecturas
						\end{tabular}
						\\ Amazon Translate & 1 millón de caracteres \\ \bottomrule
					\end{tabularx}
					\caption{Uso Mensual Estimado de los servicios AWS utilizados en el
					proyecto}
					\label{tab:uso_aws}
				\end{table}

				\begin{itemize}
					\item \textbf{AWS Lambda}:
						\[
							\text{Coste}= \frac{5 \, \text{millones}}{1 \, \text{millón}}\times
							0.20 \, \text{USD}= 1.00 \, \text{USD}
						\]

					\item \textbf{Amazon S3}:
						\begin{itemize}
							\item Solicitudes:
								\[
									\text{Coste}= \frac{500,000}{1000}\times 0.00045 \, \text{USD}=
									0.225 \, \text{USD}
								\]

							\item Almacenamiento:
								\[
									\text{Coste}= 100 \, \text{GB}\times 0.023 \, \text{USD/GB}= 2.
									30 \, \text{USD}
								\]

							\item Total:
								\[
									0.225 \, \text{USD}+ 2.30 \, \text{USD}= 2.525 \, \text{USD}
								\]
						\end{itemize}

					\item \textbf{Amazon Textract}:
						\[
							\text{Coste}= \frac{5,000}{1000}\times 1.50 \, \text{USD}= 7.50 \,
							\text{USD}
						\]

					\item \textbf{Amazon SNS}:
						\[
							\text{Coste}= \frac{2 \, \text{millones}}{1 \, \text{millón}}\times
							0.60 \, \text{USD}= 1.20 \, \text{USD}
						\]

					\item \textbf{Amazon Comprehend}:
						\[
							\text{Coste}= 5,000 \times 0.0001 \, \text{USD}= 0.50 \, \text{USD}
						\]

					\item \textbf{Amazon Lex}:
						\[
							\text{Coste}= \frac{10,000}{1000}\times 0.75 \, \text{USD}= 7.50 \,
							\text{USD}
						\]

					\item \textbf{Amazon DynamoDB}:
						\begin{itemize}
							\item Escrituras:
								\[
									\text{Coste}= \frac{3.55 \, \text{millones}}{1 \,
									\text{millón}}\times 1.25 \, \text{USD}= 4.44 \, \text{USD}
								\]

							\item Lecturas:
								\[
									\text{Coste}= \frac{3.55 \, \text{millones}}{1 \,
									\text{millón}}\times 0.25 \, \text{USD}= 0.89 \, \text{USD}
								\]

							\item Total:
								\[
									4.44 \, \text{USD}+ 0.89 \, \text{USD}= 5.33 \, \text{USD}
								\]
						\end{itemize}

					\item \textbf{Amazon Translate}:
						\[
							\text{Coste}= \frac{1 \, \text{millón}}{1 \, \text{millón}}\times 1
							5.00 \, \text{USD}= 15.00 \, \text{USD}
						\]
				\end{itemize}

			\item \underline{Google Cloud Platform}

				\begin{itemize}
					\item \textbf{Cloud Functions}: \$0.40 USD por millón de solicitudes (primeros
						2 millones gratis).

					\item \textbf{Document AI}: \$1.50 USD por 1000 páginas.

					\item \textbf{Cloud Storage}:
						\begin{itemize}
							\item Almacenamiento: \$0.023 USD por GB al mes.

							\item Operaciones: \$0.005 USD por cada 1000 operaciones.
						\end{itemize}

					\item \textbf{Cloud Tasks}: Primer millón gratis al mes.

					\item \textbf{Natural Language API}: \$0.0020 USD por unidad (1000 caracteres).

					\item \textbf{Translation API}: \$20.00 USD por millón de caracteres.

					\item \textbf{Dialogflow}: \$0.002 USD por solicitud de API.

					\item \textbf{Firestore}:
						\begin{itemize}
							\item Almacenamiento: \$0.18 USD por GB al mes.

							\item Operaciones de lectura: \$0.02 USD por 100,000 operaciones.

							\item Operaciones de escritura: \$0.18 USD por 100,000 operaciones.

							\item Operaciones de eliminación: \$0.02 USD por 100,000 operaciones.
						\end{itemize}
				\end{itemize}

				\begin{table}[h]
					\centering
					\begin{tabularx}
						{\textwidth}{@{}lX@{}} \toprule \textbf{Servicio GCP} & \textbf{Coste
						por Solicitud/Unidad/Almacenamiento} \\ \midrule Cloud Functions &
						\$0.40 USD por millón de solicitudes (primeros 2 millones gratis) \\
						Document AI & \$1.50 USD por 1000 páginas \\ Cloud Storage &
						\begin{tabular}[t]{@{}l@{}}
							• \$0.023 USD por GB al mes             \\
							• \$0.005 USD por cada 1000 operaciones
						\end{tabular}
						\\ Cloud Tasks & Primer millón gratis al mes \\ Natural Language API
						& \$0.0020 USD por unidad (1000 caracteres) \\ Translation API & \$20.00
						USD por millón de caracteres \\ Dialogflow & \$0.002 USD por solicitud
						de API \\ Firestore &
						\begin{tabular}[t]{@{}l@{}}
							• \$0.18 USD por GB al mes                          \\
							• \$0.02 USD por 100,000 operaciones de lectura     \\
							• \$0.18 USD por 100,000 operaciones de escritura   \\
							• \$0.02 USD por 100,000 operaciones de eliminación
						\end{tabular}
						\\ \bottomrule
					\end{tabularx}
					\caption{Costes por Solicitud/Unidad/Almacenamiento de los servicios
					GCP utilizados en el proyecto}
					\label{tab:costes_gcp}
				\end{table}

				Vamos a asumir las siguientes cifras de uso mensual:

				\begin{itemize}
					\item \textbf{Cloud Functions}: 3 millones de solicitudes.

					\item \textbf{Document AI}: 10,000 páginas.

					\item \textbf{Cloud Storage}:
						\begin{itemize}
							\item Operaciones: 200,000 operaciones.

							\item Almacenamiento: 200 GB.
						\end{itemize}

					\item \textbf{Cloud Tasks}: 1.5 millones de tareas.

					\item \textbf{Natural Language API}: 100,000 unidades (100 millones de
						caracteres).

					\item \textbf{Translation API}: 1 millón de caracteres.

					\item \textbf{Dialogflow}: 20,000 solicitudes de API.

					\item \textbf{Firestore}:
						\begin{itemize}
							\item Operaciones de lectura: 500,000 operaciones.

							\item Operaciones de escritura: 100,000 operaciones.

							\item Operaciones de eliminación: 100,000 operaciones.
						\end{itemize}
				\end{itemize}

				\begin{table}[h]
					\centering
					\begin{tabularx}
						{\textwidth}{@{}lX@{}} \toprule \textbf{Servicio GCP} & \textbf{Uso
						Mensual Estimado} \\ \midrule Cloud Functions & 3 millones de
						solicitudes \\ Document AI & 10,000 páginas \\ Cloud Storage &
						\begin{tabular}[t]{@{}l@{}}
							• 200,000 operaciones \\
							• 200 GB
						\end{tabular}
						\\ Cloud Tasks & 1.5 millones de tareas \\ Natural Language API &
						100,000 unidades (100 millones de caracteres) \\ Translation API & 1
						millón de caracteres \\ Dialogflow & 20,000 solicitudes de API \\
						Firestore &
						\begin{tabular}[t]{@{}l@{}}
							• 500,000 operaciones de lectura     \\
							• 100,000 operaciones de escritura   \\
							• 100,000 operaciones de eliminación
						\end{tabular}
						\\ \bottomrule
					\end{tabularx}
					\caption{Uso Mensual Estimado de los servicios GCP utilizados en el
					proyecto}
					\label{tab:uso_gcp}
				\end{table}

				A continuación, se detalla el cálculo de los costes estimados para los servicios
				de Google Cloud Platform (GCP) utilizados en el proyecto.

				\begin{itemize}
					\item \textbf{Cloud Functions}:
						\[
							\text{Coste}= \frac{3 \, \text{millones}}{1 \, \text{millón}}\times
							0.40 \, \text{USD}= 1.20 \, \text{USD}
						\]

					\item \textbf{Document AI}:
						\[
							\text{Coste}= \frac{10,000}{1000}\times 1.50 \, \text{USD}= 15.00 \,
							\text{USD}
						\]

					\item \textbf{Cloud Storage}:
						\begin{itemize}
							\item Operaciones:
								\[
									\text{Coste}= \frac{200,000}{1000}\times 0.005 \, \text{USD}= 1
									.00 \, \text{USD}
								\]

							\item Almacenamiento:
								\[
									\text{Coste}= 200 \, \text{GB}\times 0.023 \, \text{USD/GB}= 4.
									60 \, \text{USD}
								\]

							\item Total:
								\[
									1.00 \, \text{USD}+ 4.60 \, \text{USD}= 5.60 \, \text{USD}
								\]
						\end{itemize}

					\item \textbf{Cloud Tasks}:
						\[
							\text{Coste}= \frac{1.5 \, \text{millones}}{1 \, \text{millón}}\times
							0.00 \, \text{USD}= 0.00 \, \text{USD}
						\]

					\item \textbf{Natural Language API}:
						\[
							\text{Coste}= \frac{100,000}{1000}\times 0.002 \, \text{USD}= 0.20
							\, \text{USD}
						\]

					\item \textbf{Translation API}:
						\[
							\text{Coste}= \frac{1 \, \text{millón}}{1 \, \text{millón}}\times 2
							0.00 \, \text{USD}= 20.00 \, \text{USD}
						\]

					\item \textbf{Dialogflow}:
						\[
							\text{Coste}= \frac{20,000}{1,000}\times 0.002 \, \text{USD}= 0.04
							\, \text{USD}
						\]

					\item \textbf{Firestore}:
						\begin{itemize}
							\item Operaciones de lectura:
								\[
									\text{Coste}= \frac{500,000}{100,000}\times 0.02 \, \text{USD}=
									0.10 \, \text{USD}
								\]

							\item Operaciones de escritura:
								\[
									\text{Coste}= \frac{100,000}{100,000}\times 0.18 \, \text{USD}=
									0.18 \, \text{USD}
								\]

							\item Operaciones de eliminación:
								\[
									\text{Coste}= \frac{100,000}{100,000}\times 0.02 \, \text{USD}=
									0.02 \, \text{USD}
								\]

							\item Almacenamiento:
								\[
									\text{Coste}= 1 \, \text{GB}\times 0.18 \, \text{USD/GB}= 0.18
									\, \text{USD}
								\]

							\item Total:
								\[
									0.10 \, \text{USD}+ 0.18 \, \text{USD}+ 0.02 \, \text{USD}+ 0.1
									8 \, \text{USD}= 0.48 \, \text{USD}
								\]
						\end{itemize}
				\end{itemize}
		\end{enumerate}

		\begin{table}[h]
			\centering
			\begin{tabularx}
				{\textwidth}{@{}lXr@{}} \toprule \textbf{Concepto} & \textbf{Descripción}
				& \textbf{Coste Mensual (USD)} \\ \midrule \textbf{Licencia Windows 11}
				& Coste de la licencia de Windows 11 Pro & 13.20 \\ \midrule \textbf{Servicios
				AWS} & & \\ AWS Lambda & 5 millones de solicitudes & 1.00 \\ Amazon S3 &
				Almacenamiento y solicitudes & 2.53 \\ Amazon Textract & 5,000 páginas procesadas
				& 7.50 \\ Amazon SNS & 2 millones de notificaciones & 1.20 \\ Amazon Comprehend
				& 5,000 unidades & 0.50 \\ Amazon Lex & 10,000 solicitudes de texto & 7.50
				\\ Amazon DynamoDB & Escrituras y lecturas & 5.33 \\ Amazon Translate & 1
				millón de caracteres & 15.00 \\ \midrule \textbf{Servicios GCP} & & \\
				Cloud Functions & 3 millones de solicitudes & 1.20 \\ Document AI & 10,000
				páginas & 15.00 \\ Cloud Storage & Almacenamiento y operaciones & 5.60
				\\ Cloud Tasks & 1.5 millones de tareas & 0.00 \\ Natural Language API &
				100,000 unidades & 0.20 \\ Translation API & 1 millón de caracteres &
				20.00 \\ Dialogflow & 20,000 solicitudes de API & 0.04 \\ Firestore &
				Operaciones y almacenamiento & 0.48 \\ \midrule \textbf{Total (mensual)}
				& & \textbf{96.28} \\ \bottomrule
			\end{tabularx}
			\caption{Coste total mensual del proyecto, combinando los servicios de AWS,
			GCP y la licencia de Windows 11}
			\label{tab:coste_total}
		\end{table}

		Para obtener los gastos totales del proyecto, simplemente se multiplica el
		costo mensual por el número de meses:

		\[
			\text{{Gastos totales}}= 96,28 \, \text{{€}}\times 4 \,\text{{ meses}}\approx
			385,12 \, \text{{€}}
		\]
\nota{Se considera un escenario en el que la empresa se contratan ambas plataformas. Sin embargo, en realidad esto no es así, se debería elegir entre una u otra dividiendo el coste de las licencias a la mitad.}
	\item \underline{Costes de \textit{hardware}}

		Para calcular los costes de \textit{hardware}, se considerará el portátil del
		alumno con el que ha realizado el proyecto. Este se trata de ASUS ROG Strix
		G15 con un procesadorAMD Ryzen 9 4.9GHz y una RAM de 32 GB. Este portátil se
		encuentra en el mercado a un precio de 1.500 €.

		El coste de amortización se basará en su vida útil estimada. Supongamos unos
		4 años, para calcular el coste anual de amortización, dividiremos el coste del
		portátil entre el número de años de vida útil:

		\[
			\text{{Coste anual de amortización}}= \frac{{1.500 \, \text{{€}}}}{{4 \, \text{{años}}}}
			\approx 375\, \text{{€/año}}
		\]

		Por tanto, como el proyecto ha durado 4 meses, los costes de \textit{hardware}
		suponen 302,95 por 33,33\% (proporción de los 4 meses sobre un año).

		\[
			\text{{Costes de \textit{hardware}}}= 375 \,\text{{€}}\, \times \, 33,33\%
			\approx 124,99 \, \text{{€}}
		\]
		\newline

	\item \underline{Costes de los empleados}.

		Se procederá al cálculo del salario bruto de cada integrante del equipo de
		acuerdo con la normativa laboral vigente en España para un proyecto de
		desarrollo de \textit{software}. El salario bruto se determina considerando
		factores tales como la categoría profesional y la experiencia de cada
		empleado.

		\begin{enumerate}
			\item Desarrollador: El alumno es considerado un programador \textit{junior}.
				En España, un programador \textit{junior} (con menos de 3 años de experiencia
				laboral) puede obtener una remuneración de alrededor de 19\,700 € brutos
				por año~\cite{Jobted}. Además, según el artículo 34 del Estatuto de los
				Trabajadores, la duración fijada en convenio colectivo de la jornada laboral
				es un máximo de 40 horas de trabajo efectivo~\cite{jornadalaboral}.
				Ahora, para calcular el salario bruto mensual, se puede utilizar la
				siguiente fórmula\footnote{Entendemos habitualmente 14 pagas (12 meses y
				2 pagas extra).}:

				\[
					\text{{Salario Bruto Mensual}}= \frac{{\text{{Salario Bruto Anual}}}}{{14 \, \text{{pagas}}}}
				\]

				Sustituyendo el valor del salario bruto anual, obtenemos:

				\[
					\text{{Salario Bruto Mensual}}= \frac{{19\,700 \,\text{{€}}}}{{14 \,\text{{pagas}}}}
					\approx 1\,407,14 \, \text{{€}}
				\]

			\item \underline{Product Owner}: el salario bruto promedio anual para un Product
				Owner en España es de 41.866 €~\cite{PayScale}. Utilizando la misma
				fórmula para calcular el salario bruto mensual, obtenemos:

				\[
					\text{{Salario Bruto Mensual}}= \frac{{41\,866 \, \text{{€}}}}{{14 \, \text{{pagas}}}}
					\approx 2\,990,43 \,\text{{€}}
				\]

			\item \underline{Scrum Master}: el salario bruto promedio anual para un Scrum
				Master en España es de 40.626 €~\cite{PayScale1}. Utilizando la misma
				fórmula para calcular el salario bruto mensual, obtenemos:

				\[
					\text{{Salario Bruto Mensual}}= \frac{{40\,626 \,\text{{€}}}}{{14 \, \text{{pagas}}}}
					\approx 2\,901,85 \,\text{{€}}
				\]
		\end{enumerate}

		Una vez obtenido el salario bruto, se deben aplicar los impuestos
		correspondientes de acuerdo con la legislación fiscal vigente. Los impuestos
		a tener en cuenta incluyen las contribuciones a la Seguridad Social~\cite{Cotización},
		el impuesto sobre la renta\footnote{En cuanto al IRPF, el porcentaje aplicado
		puede variar según el nivel de ingresos y la situación fiscal personal.
		Según la Agencia Tributaria de España, se pueden encontrar las tablas y los porcentajes
		correspondientes en su documentación oficial~\cite{AEAT}.} (IRPF)~\cite{IRPF}
		y otros impuestos relacionados con la contratación laboral. La información
		detallada sobre los impuestos y las obligaciones legales se encuentra en los
		documentos oficiales y normativas correspondientes proporcionados por las
		autoridades fiscales y laborales citados anteriormente.

		De forma resumida, Para el año 2024, se han establecido los siguientes
		porcentajes\footnote{El gobierno pone a disposición la plataforma ipyme~\cite{ipyme},
		que proporciona información valiosa para facilitar el cálculo de los tramites
		empresariales.}, los cuales se detallan en la Tabla
		\ref{tab:seguridad-social} y Tabla \ref{tab:no-seguridad-social}.

		\begin{table}[h]
			\centering
			\begin{tabularx}
				{320}{@{}XXr@{}} \toprule \textbf{Concepto} & \textbf{Descripción} &
				\textbf{Porc. (\%)} \\ \midrule Cuota Contingencias Comunes & Gastos comunes
				en el ámbito laboral & 23,60 \\ Cuota Formación Profesional & Inversión en
				formación y capacitación & 0,60 \\ Desempleo & Prestaciones por desempleo
				& 5,50 \\ Accidentes de Trabajo & Prevención y compensación por accidentes
				laborales & 1,50 \\ FOGASA & Fondo de Garantía Salarial & 0,20 \\ \bottomrule
			\end{tabularx}
			\caption{Conceptos de Seguridad Social}
			\label{tab:seguridad-social}
		\end{table}

		\begin{table}[h]
			\centering
			\begin{tabularx}
				{320}{@{}XXr@{}} \toprule \textbf{Concepto} & \textbf{Descripción} &
				\textbf{Porcentaje} \\ \midrule IRPF & Impuesto sobre la Renta de las Personas
				Físicas & Variable\\ IVA & Impuesto sobre el Valor Añadido (tipo general)
				& 21,00\% \\ \bottomrule
			\end{tabularx}
			\caption{Otros conceptos}
			\label{tab:no-seguridad-social}
		\end{table}

		Con base en los cálculos previos de los salarios brutos y los impuestos
		aplicados, se puede determinar el coste total de la empresa en cuanto a los empleados.

		\begin{enumerate}
			\item \underline{Desarrollador}: Dado que el salario bruto mensual del desarrollador
				es de aproximadamente 1\,641,67 €, se aplicarán los porcentajes
				correspondientes de los conceptos de la Seguridad Social según lo
				establecido en la Tabla \ref{tab:seguridad-social}. El total de los conceptos
				de Seguridad Social se calcula sumando los porcentajes aplicados al salario
				bruto mensual:

				\begin{align*}
					\text{Total SS}= 0.236 + 0.006 + 0.055 + 0.015 + 0.002 \approx 0.314
				\end{align*}

				El IRPF es un impuesto variable y su porcentaje de retención en nómina
				depende del nivel de ingresos y la situación fiscal personal. Por lo tanto,
				no se tendrá en cuenta para el resultado de este cálculo.

				Por último, el gasto de la empresa se calcula dividiendo el salario bruto
				mensual entre 1 menos el total de la Seguridad Social. Sustituyendo los
				valores correspondientes, obtenemos:

				\[
					\text{{Gasto mensual}}= \frac{{1\,407,14 \, \text{{€}}}}{{1 - 0,314 }}\approx
					2\,051,22 \, \text{{€}}
				\]

			\item \underline{Product Owner}: Como el salario bruto anual para el Product
				Owner es de 3\,488,83 €, aplicando los porcentajes de la Tabla
				\ref{tab:seguridad-social}, el gasto total de la empresa para el Product
				Owner sería:

				\[
					\text{{Salario mensual}}= \frac{{2\,990,43 \text{{€}}}}{{1 - 0,314}}\approx
					4\,359,22 \, \text{{€}}
				\]
				De las 154 horas de la jornada mensual (22 días laborables por 7 horas
				al día), hemos calculado una dedicación del 7\% de su tiempo. Por tanto,
				se va aplicar este 7\% al gasto mensual del Product Owner y del Scrum
				Master. Se ha estimado una dedicación de 2 horas de reuniones semanales más
				3 horas repartidas a lo largo del mes.

				\[
					\text{{Gasto mensual}}= 4\,359,22 \, \text{{€}}\times 0,07 \approx 305,
					13 \,\text{{€}}
				\]

			\item \underline{Scrum Master}: De manera similar, para el Scrum Master, partimos
				de un salario bruto mensual aproximado de 3\,385,50 €. Aplicando los
				porcentajes de la Tabla \ref{tab:seguridad-social}, el gasto total de la
				empresa para el Scrum Master sería:

				\[
					\text{{Salario mensual}}= \frac{{2\,901,85 \, \text{{€}}}}{{1 - 0,314 }}
					\approx 4\,230,10 \, \text{{€}}
				\]

				Igualmente, al Scrum Master se le aplica un coeficiente reductor del 7\%
				de su salario, dado que ha invertido el mismo tiempo que el Product Owner.

				\[
					\text{{Gasto mensual}}= 4\,230,10 \, \text{{€}}\times 0,07 \approx 296,
					11 \,\text{{€}}
				\]
		\end{enumerate}

		A modo de resumen, se agrupa el resultado de todos los cálculos en la Tabla \ref{tab:costes-empleados}.

		\begin{table}[h]
			\centering
			\begin{tabularx}
				{300}{@{}Xr@{}} \toprule \textbf{Empleado} & \textbf{Gasto Mensual (€)}\\
				\midrule Desarrollador & 2\,051,22 \\ Product Owner & 305,13 \\ Scrum Master
				& 296,11 \\ \midrule \textbf{Total Proyecto} & 2\,652,46 \\ \bottomrule
			\end{tabularx}
			\caption{Resumen de costes del personal}
			\label{tab:costes-empleados}
		\end{table}
\end{enumerate}

Para visualizar los costes de forma clara, se ha elaborado una última tabla resumen con la suma de los costes finales de software, hadware y personal: (ver Tabla \ref{tab:costes-totales}).
\nota{Se ha hecho una conversión de dólares a euros, considerando que un dólar son 0,92 euros, para las licencias software. El precio total de las licencias era de 385 dólares, que son 355 euros.}

\begin{table}[h]
\centering
\begin{tabularx}{\textwidth}{@{}lXr@{}}
\toprule
\textbf{Concepto} & \textbf{Descripción} & \textbf{Coste Total (€)} \\
\midrule
Personal & Costes totales de empleados & 10,609.84 \\
Hardware & Costes por el portátil & 124.99 \\
Software & Costes totales por licencias y servicios & 355 \\
\midrule
\textbf{Coste Total del Proyecto} & & \textbf{11,089,83} \\
\bottomrule
\end{tabularx}
\caption{Resumen de costes del proyecto}
\label{tab:costes-totales}
\end{table}


\subsubsection{Beneficios}

A pesar de que el fin de la aplicación no sea lucrativos- estará disponible de forma gratuita para la comunidad científica -es importante considerar una forma hipotética de generar ingresos para sostener el proyecto. Valoraremos diferentes opciones.

\begin{enumerate}
    \item \underline{Donaciones y Financiación Pública}:
    \begin{itemize}
        \item \textbf{Crowdfunding}: Se puede lanzar una campaña de financiación colectiva a través de plataformas como Kickstarter, Indiegogo, o GoFundMe. Este método permite a los usuarios interesados en la aplicación contribuir económicamente para su sostenimiento y futuras mejoras.
        \item \textbf{Subvenciones y Becas}: Existen diversas organizaciones y entidades gubernamentales que ofrecen subvenciones y becas para proyectos de investigación y desarrollo tecnológico. Aplicar a estas ayudas puede proporcionar los fondos necesarios para mantener el proyecto.
        \item \textbf{Donaciones Directas}: Habilitar una opción de donaciones directas en la página web del proyecto puede permitir a los usuarios hacer contribuciones voluntarias mediante plataformas como PayPal, Patreon, o BuyMeACoffee puede facilitar este proceso.
    \end{itemize}

    \item \underline{Servicios de Consultoría y Personalización}:
    \begin{itemize}
        \item \textbf{Consultoría Técnica}: Ofrecer servicios de consultoría técnica a organizaciones que necesiten asesoramiento especializado en la implementación y uso de chatbots. 
    \end{itemize}

    \item \underline{Publicidad y Patrocinios}:
    \begin{itemize}
        \item \textbf{Publicidad en la Plataforma}: Implementar espacios publicitarios dentro de la aplicación y en su página web. Esto puede generar ingresos a través de la venta de estos espacios a empresas interesadas en llegar a la comunidad científica.
        \item \textbf{Patrocinios}: Establecer acuerdos de patrocinio con empresas y organizaciones relacionadas con el campo científico y tecnológico. Los patrocinadores pueden aportar fondos a cambio de la visibilidad de su marca en la plataforma.
    \end{itemize}

    \item \underline{Modelos de Suscripción}:
    \begin{itemize}
        \item \textbf{Freemium}: Ofrecer una versión básica gratuita de la aplicación con funcionalidades limitadas y una versión premium con características avanzadas mediante una suscripción mensual o anual.
        \item \textbf{Suscripción a Contenidos Exclusivos}: Proporcionar acceso a contenidos exclusivos, como informes detallados, análisis avanzados, y herramientas especializadas, a través de una suscripción.
    \end{itemize}

    \item \underline{Colaboraciones y Alianzas Estratégicas}:
    \begin{itemize}
        \item \textbf{Alianzas con Universidades y Centros de Investigación}: Establecer colaboraciones con universidades y centros de investigación para el uso y promoción de la aplicación en sus proyectos y programas educativos.
        \item \textbf{Proyectos Conjuntos}: Colaborar con otras organizaciones y empresas en proyectos conjuntos que utilicen la aplicación, compartiendo tanto los costos como los beneficios generados.
    \end{itemize}
\end{enumerate}

\subsection{Viabilidad legal}\label{sec:Viabilidad legal}

La viabilidad legal implica evaluar las cuestiones jurídicas y normativas que puedan afectar la operación y distribución de la aplicación \textit{open source}. Es importante asegurarse de cumplir con las regulaciones y requisitos legales relacionados con la privacidad, protección de datos y derechos de propiedad intelectual.

\subsubsection{Descargo de responsabilidades}

El \emph{software} que se proporciona en este trabajo es solo para fines informativos y debe tenerse cuidado con los servicios que se utilizan dado que pueden generar gastos excesivos. 

\subsubsection{Licencias \emph{software}}
La única licencia que se ha utilizado para el desarrollo del proyecto ha sido python, la cual es OSI-open source. 
\capitulo{6}{Trabajos relacionados}

En el ámbito del desarrollo de chatbots, existen numerosos trabajos que se centran en la creación de soluciones sencillas utilizando Lex o DialogFlow. Sin embargo, muchos de estos trabajos se enfocan en un subconjunto específico de servicios y no integran de manera completa todas las capacidades disponibles en AWS. Esta fragmentación de soluciones ha llevado a una necesidad de agrupar y combinar varias implementaciones para lograr un sistema robusto y completo, como el desarrollado en este trabajo.

Por ejemplo, un estudio sobre la creación de un asistente virtual interactivo para la programación utilizando Amazon Lex y Lambda proporciona una pequeña guía sobre cómo construir un chatbot con estos servicios. Sin embargo, este trabajo no integra otros servicios esenciales como Amazon Textract, Amazon Comprehend, o Amazon S3 \citep{online:lex_lambda_c}.

Otro ejemplo, diría que el más relevante, ha sido el proyecto de github sobre la extracción de \textit{insights} de facturas con Amazon Textract, Amazon Comprehend y Amazon Lex, muestra cómo utilizar estos servicios específicos para automatizar el procesamiento de datos textuales y descubrir insights a partir de ellos \citep{online:github_textract}. Si bien este proyecto integra un mayor número de servicios para el chatbot, su funcionalidad es muy limitada proporcionando únicamente datos relevantes de facturas.

Sin embargo, la documentación proporcionada por AWS ha sido lo suficientemente detallada para guiarme en la creación e integración de todos estos servicios \citep{online:aws_docs}. Los tutoriales y guías oficiales de AWS ofrecen ejemplos claros y pasos detallados para implementar cada servicio.

Por el otro lado, el caso de GCP ha sido muy similar. Sin embargo, el principal problema ha sido la complejidad de la documentación que proporciona google, ya que tiende a ser más compleja y extensa \citep{online:google_docs}. A pesar de este obstáculo, una vez que se comprendió el funcionamiento de los servicios de AWS, la transición y aplicación de conocimientos a GCP resultó ser mucho menos complicada.

Por ejemplo, un trabajo sobre la creación de un chatbot utilizando Dialogflow y Google Cloud Functions sería el siguiente, donde se explica la creación y configuración de Dialogflow, muy similar a Amazon Lex \citep{online:gcp_chatbot}.
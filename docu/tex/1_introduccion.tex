\chapter{Introducción}

% Texto introductorio
En la era digital actual, las empresas están constantemente buscando formas de mejorar la eficiencia y la experiencia del cliente. Los chatbots han surgido como la herramienta esencial capaz de lograr sus objetivos. Desde su aparición en el siglo pasado, los chatbots han evolucionado significativamente, pasando de simples scripts de respuesta a sistemas avanzados de inteligencia artificial capaces de comprender y generar lenguaje natural, como el ofrecido por la empresa OpenAI.

Las empresas que implementan estas tecnologías no sólo tienen como objetivo mejorar la interacción con el cliente sino también optimizar los procesos internos y reducir los gastos operativos. Los chatbots se han utilizado en muchas industrias, incluida la atención al cliente, el comercio electrónico, la atención médica y muchas más.
 Según un estudio de McKinsey & Company, el uso de chatbots puede reducir los costos de atención al cliente en hasta un 30\%, al mismo tiempo que mejora la resolución de problemas en primera instancia en un 20\% (McKinsey, 2020).

 Siguiendo las recomendaciones de expertos en la industria, como las proporcionadas por Gartner, las empresas deben considerar varios factores al elegir una plataforma de chatbot, incluyendo la capacidad de integración, la escalabilidad y el soporte técnico (Gartner, 2022). 
 

\section{Estructura de la memoria}\label{estructura-de-la-memoria}

La memoria sigue la siguiente estructura:

\begin{itemize}
\tightlist
\item
  \textbf{Introducción:} breve descripción del problema a resolver y la
  solución propuesta. Estructura de la memoria y listado de materiales
  adjuntos.
\item
  \textbf{Objetivos del proyecto:} exposición de los objetivos que
  persigue el proyecto.
\item
  \textbf{Conceptos teóricos:} breve explicación de los conceptos
  teóricos clave para la comprensión de la solución propuesta.
\item
  \textbf{Técnicas y herramientas:} listado de técnicas metodológicas y
  herramientas utilizadas para gestión y desarrollo del proyecto.
\item
  \textbf{Aspectos relevantes del desarrollo:} exposición de aspectos
  destacables que tuvieron lugar durante la realización del proyecto.
\item
  \textbf{Trabajos relacionados:} estado del arte en las aplicaciones y sitios web de bolsa y finanzas.
\item
  \textbf{Conclusiones y líneas de trabajo futuras:} conclusiones
  obtenidas tras la realización del proyecto y posibilidades de mejora o
  expansión de la solución aportada.
\end{itemize}
Junto a la memoria se proporcionan los siguientes anexos:

\begin{itemize}
\tightlist
\item
  \textbf{\\todo} \\todo
\end{itemize}

\section{Materiales adjuntos}\label{materiales-adjuntos}

Los materiales que se adjuntan con la memoria son: 

\begin{itemize}
\item
	\textbf{Anexos}: consultar \href{https://github.com/rmt0009alu/FAT/blob/main/docs/latex/anexos.pdf}{la documentación técnica}.

\end{itemize}

Además, los siguientes recursos están accesibles a través de internet:

\begin{itemize}
\item
  \textbf{Repositorio}: visitar \href{}{GitHub}.
\end{itemize}



